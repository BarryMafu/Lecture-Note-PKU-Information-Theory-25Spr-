\chapter{Lecture 13}

%--- 信息 ----
\begin{center}
    讲师:王立威 \qquad
    课程时间:25.May.20th \qquad 
    笔记:25.June.9th
\end{center}

\bigskip

首先完成了信道编码定理的证明,放在了前一节. 下面介绍Fisher信息和Carmer-Rao不等式. 

动机和来源是无偏估计. (若学过统计学可跳过)
\begin{definition}[无偏估计]
    我们(往往独立同分布地)采样了一组样本点$X=(X_1,X_2,\dots, X_n)$,设其密度函数为$f(\cdot;\te)$(其中$\te$是我们要估计的参数),那么有 
    \[
    f(x;\te) = \prod_{i=1}^n f(X_i; \te)
    \]
    
    现在使用这些样本点得到$\te$的一个估计$\hat{\te} = \phi(X_1,\dots, X_n)$. 如果满足$\E[\hat{\te}] = \te$则称$\hat{\te}$是$\te$的\textbf{无偏估计}(unbiased estimation).
\end{definition}


% \begin{figure}[H]
%     \centering
%     \includegraphics[width=.6\textwidth]{images/c2_1.png}
%     \caption{$H=x\log 1/x + (1-x)\log 1/(1-x)$的图像}
% \end{figure}